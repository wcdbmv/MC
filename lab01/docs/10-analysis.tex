%% Методические указания к выполнению, оформлению и защите выпускной квалификационной работы бакалавра
%% 2.4 Аналитический раздел
%%
%% В данном разделе расчётно-пояснительной записки проводится анализ предметной области и выделяется основной объект исследования.
%% Если формализовать предметную область с помощью математической модели не удаётся и при этом она сложна для понимания, то для отображения происходящих в ней процессов необходимо использовать методологию IDEF0, а для описания сущностей предметной области и взаимосвязей между ними — ER-модель.
%%
%% Затем выполняется обзор существующих методов и алгоритмов решения идентифицированной проблемы предметной области (опять же с обязательными ссылками на научные источники: монографии, статьи и др.) и их программных реализаций (при наличии), анализируются достоинства и недостатки каждого из них.
%% Выполненный обзор должен позволить объективно оценить актуальное состояние изучаемой проблемы.
%% Результаты проведённого анализа по возможности классифицируются и оформляются в табличной форме.
%%
%% На основе выполненного анализа обосновывается необходимость разработки нового или адаптации существующего метода или алгоритма.
%%
%% Если же целью анализа являлся отбор (на основе чётко сформулированных критериев) тех методов и алгоритмов, которые наиболее эффективно решают поставленную задачу, то форма представления результата должна подтвердить обоснованность сделанного выбора, в том числе — полноту и корректность предложенных автором критериев отбора.
%%
%% Одним из основных выводов аналитического раздела должно стать формализованное описание проблемы предметной области, на решение которой будет направлен данный проект, включающее в себя:
%% — описание входных и выходных данных;
%% — указание ограничений, в рамках которых будет разработан новый, адаптирован существующий или просто реализован метод или алгоритм;
%% — описание критериев сравнения нескольких реализаций метода или алгоритма;
%% — описание способов тестирования разработанного, адаптированного или реализованного метода или алгоритма;
%% — описание функциональных требований к разрабатываемому программному обеспечению,
%% при этом в зависимости от направления работы отдельные пункты могут отсутствовать.
%%
%% Если в результате работы будет создано программное обеспечение, реализующее большое количество типичных способов взаимодействия с пользователем, необходимо каждый из этих способов описать с помощью диаграммы прецедентов [4, 5].
%%
%% Рекомендуемый объём аналитического раздела 25—30 страниц.


\chapter{Аналитический раздел}

\section{Постановка задачи}

\subsection{Содержательная постановка}

В распоряжении работодателя имеется $n$ работ и $n$ исполнителей.
Стоимость выполнения $i$ работы $j$ исполнителем составляет $c_{ij} > 0$ единиц.

\begin{itemize}
	\item Требуется распределить все работы по исполнителям так, чтобы каждый исполнитель выполнял ровно 1 работу.
	\item Общая стоимость всех работ должна быть минимальной.
\end{itemize}

\subsection{Математическая постановка}

Обозначим за матрицу стоимостей
\begin{equation}
	C = (c_{ij}), \quad i,j=\overline{1,n}
\end{equation}

Введём так называемые управляемые переменные:
\begin{equation}
	x_{ij} = \begin{cases}
		1, \text{ если } i \text{-ю работу выполняет } j \text{-й работник}; \\
		0, \text{ иначе}.
	\end{cases}
	i, j = \overline{1, n}
\end{equation}

Обозначим за матрицу назначений
\begin{equation}
	X = (x_{ij}), \quad  i, j = \overline{1, n}.
\end{equation}

Математическая постановка задачи о назначениях:
\begin{equation}
	\begin{dcases}
		f(x) = \sum_{i=1}^{n} \sum_{j=1}^n c_{ij}x_{ij} \to \min, \\
		\sum_{i=1}^{n} x_{ij} = 1, \quad j =\overline{1, n}, \\
		\sum_{j=1}^{n} x_{ij} = 1, \quad i =\overline{1, n}, \\
		x_{ij} \in \{0, 1\}, \quad i, j = \overline{1, n}. \\
	\end{dcases}
\end{equation}

\section{Исходные данные варианта 6}

\begingroup
\renewcommand*{\arraystretch}{0.6}
\begin{equation}
	C = \begin{bmatrix}
		10 &  8 &  6 &  4 &  9 \\
		11 &  9 & 10 &  5 &  6 \\
		 5 & 10 &  8 &  6 &  4 \\
		 3 & 11 &  9 &  6 &  6 \\
		 8 & 10 & 11 &  8 &  7 \\
	\end{bmatrix}
\end{equation}
\endgroup
