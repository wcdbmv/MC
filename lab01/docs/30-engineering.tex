%% Методические указания к выполнению, оформлению и защите выпускной квалификационной работы бакалавра
%% 2.6 Технологический раздел
%%
%% Технологический раздел содержит обоснованный выбор средств программной реализации, описание основных (нетривиальных) моментов разработки и методики тестирования созданного программного обеспечения.
%%
%% В этом же разделе описывается информация, необходимая для сборки и запуска разработанного программного обеспечения, форматы входных, выходных и конфигурационных файлов (если такие имеются), а также интерфейс пользователя и руководство пользователя.
%%
%% Если для правильного функционирования разработанного программного обеспечения требуется некоторая инфраструктура (веб-приложение, база данных, серверное приложение), уместно представить её с помощью диаграммы развёртывания UML.
%%
%% Как уже говорилось, часть технологического раздела должна быть посвящена тестированию разработанного программного обеспечения.
%%
%% Модульное тестирование описывается в технологическом разделе.
%%
%% Системное тестирование может быть описано в технологическом или экспериментальном разделах, в зависимости от глубины его реализации и тематики бакалаврской работы.
%%
%% При проведении тестирования разработанного программного обеспечения следует широко использовать специализированные программные приложения: различные статические анализаторы кода (например, clang); для тестирования утечек памяти в языках программирования, где отсутствует автоматическая «сборка мусора», Valgrind, Doctor Memory и их аналоги, и т. п.
%%
%% Рекомендуемый объём технологического раздела 20—25 страниц.


\chapter{Технологический раздел}

\section{Листинг программы}

\lstinputlisting[
	caption={\code{lab01.m}},
	label={lst:lab01m},
]{../lab01.m}
