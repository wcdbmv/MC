%% Методические указания к выполнению, оформлению и защите выпускной квалификационной работы бакалавра
%% 2.5 Конструкторский раздел
%%
%% В конструкторском разделе описывается разрабатываемый и/или модифицируемый метод или алгоритм.
%%
%% В случае если в бакалаврском проекте разрабатывается новый метод или алгоритм, необходимо подробно изложить их суть, привести всё необходимые для их реализации математические выкладки, обосновать последовательность этапов выполнения.
%% При этом для каждого этапа следует выделить необходимые исходные данные и получаемые результаты.
%%
%% При использовании известного алгоритма следует указать специфические особенности его практической реализации, присущие решаемой задаче, и пути их решения в ходе программирования.
%% Для описания метода или алгоритма необходимо выбрать наиболее подходящую форму записи (схема (ГОСТ 19.701-90), диаграмма деятельности, псевдокод и т. п.).
%% Учитывая, что на эффективность алгоритма непосредственно влияют используемые структуры данных, в данном разделе РПЗ целесообразно провести сравнительный анализ структур, которые могут быть применены в рамках программной реализации выбранного алгоритма, и обосновать выбор одной из них.
%% В конце описания разработанного и/или модифицируемого алгоритма должны быть приведены выбранные способы тестирования и сами тесты.
%%
%% Перед формированием тестовых наборов данных целесообразно указать выделенные классы эквивалентности.
%% В данной части расчётно-пояснительной записки могут также выполняться расчёты для определения объёмов памяти, необходимой для хранения данных, промежуточных и окончательных результатов работы программы, а также расчёты, позволяющие оценить время решения задачи на ЭВМ.
%% Эти результаты могут использоваться для обоснования правильности выбора метода и/или алгоритма из имеющихся альтернативных вариантов, а также для оценки возможности практически реализовать поставленную задачу на имеющейся технической базе.
%%
%% Другой важный момент, который должен найти своё отражение в конструкторском разделе, это описание структуры разрабатываемого программного обеспечения.
%% Обычно оно включает в себя:
%% — описание общей структуры — определение основных частей (компонентов) и их взаимосвязей по управлению и по данным;
%% — декомпозицию компонентов и построение структурных иерархий;
%% — проектирование компонентов.
%%
%% Для графического представления такого описания, если есть необходимость, следует использовать:
%% — функциональную модель IDEF0 с декомпозицией решения исходной задачи на несколько уровней (разрабатываемые модули обычно играют роль механизмов);
%% — спецификации компонентов (процессов);
%% — модель данных (ER-диаграмма);
%% — диаграмму классов;
%% — диаграмму компонентов;
%% — диаграмму переходов состояний (конечный автомат), характеризующих поведение системы во времени.
%%
%% Рекомендуемый объем конструкторского раздела 25—30 страниц.


\chapter{Конструкторский раздел}

\section{Краткое описание венгерского метода}

\begin{algorithm}[H]
	\caption{Венгерский метод решения задачи о назначениях}
	\label{lst:euclidean-clustering}
	\small
	\begin{algorithmic}[1]
		\State \textbf{Начало}
		\State Из каждого столбца матрицы назначения вычитаем его min элемент \label{prepare-begin}
		\State Из каждой строки матрицы назначения вычитаем её min элемент
		\State Строим начальную СНН: просм. ст-цы тек. м-цы ст-тей (в порядке возр-я номера ст-ца) сверху вниз.
		Первый в ст-це нуль, в одной стр. с кот. нет 0*, отмечаем 0* \label{prepare-end}
		\State $k \coloneqq |\text{СНН}|$ \label{main-begin}
		\If{$k = n$} \label{label1}
			\State Записываем оптимальное решение:
			\begin{equation}
				x_{ij}^* \coloneqq \begin{cases}
					1, \text{ если в позиции } $(i, j)$ \text{ м-цы ст-тей стоит 0*,}\\
					0, \text{ иначе}
				\end{cases}
			\end{equation}
			\State $f^* \coloneqq f(X^*)$
			\State Вывод $X^*, f^*$
			\State \textbf{Конец}
		\Else
			\State Столбцы с 0* отмечаем "+"
			\If{Среди невыделенных элементов есть 0} \label{label2}
				\State Отмечаем его 0'
				\If{В одной строке с текущим 0' есть 0*}
					\State{Снимаем выделение со столбца с этим 0*, выделяем "+" стр. с тек. 0'}
					\State \Goto{label2}
				\Else
					\State Строим непродолж. $L$-цеп.: от тек. 0' по ст-цу в 0* по стр. $\ldots$ по стр. в 0'
					\State В пределах $L$-цепочки: $0^* \longmapsto 0; \;\; 0' \longmapsto 0^*$
					\State Снимаем все выделения, $k \coloneqq |\text{СНН}|$
					\State \Goto{label1}
				\EndIf
			\Else
				\State Ищем $h$ — min элемент среди невыделенных
				\State Вычитаем $h$ из невыд. столбцов, добавляем $h$ к выд. стркокам.
				\State \Goto{label2}
			\EndIf
		\EndIf \label{main-end}
	\end{algorithmic}
\end{algorithm}

Шаги алгоритма с \ref{prepare-begin} по \ref{prepare-end} называются подготовительным этапом, с \ref{main-begin} по \ref{main-end} — основным этапом.

Задача о назначениях для максимизации стоимости сводится к существующему алгоритму минимизации стоимости заменой целевой функции на
\begin{equation}
	f_2(x) = \sum_{i=1}^{n} \sum_{j=1}^n (M - c_{ij})x_{ij} \to \min,
\end{equation}
где $\displaystyle M = \max_{i, j = \overline{1, n}}\{ c_{ij} \}$.
