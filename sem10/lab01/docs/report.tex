\documentclass{bmstu-gost-7-32}

\begin{document}

\makereporttitle
	{Информатика, искусственный интеллект и системы управления} % Название факультета
	{Программное обеспечение ЭВМ и информационные технологии} % Название кафедры
	{лабораторной работе №~1} % Название работы (в дат. падеже)
	{Методы вычислений} % Название курса (необязательный аргумент)
	{Метод поразрядного поиска} % Тема работы
	{6} % Номер варианта (необязательный аргумент)
	{ИУ7-23М} % Номер группы
	{Керимов~А.~Ш.} % ФИО студента
	{Власов~П.~А.} % ФИО преподавателя

\section*{Постановка задачи}

Решить одномерную задачу оптимизации вида
\begin{equation}
	\begin{cases}
		f(x) \to \min, \\
		x \in [a, b],
	\end{cases}
\end{equation}
методом поразрядного поиска с заданной точностью $\varepsilon$.

\section*{Входные данные}

Заданная функция:
\begin{equation}
	f(x) = \ch{\left(\frac{3x^3 + 2x^2 - 4x + 5}{3}\right)} + \th{\left(\frac{x^3 - 3\sqrt2x - 2}{2x + \sqrt2}\right)} - 2,5.
\end{equation}

Поиск точки минимума производится на отрезке $[0, 1]$.
При построении таблицы результатов в качестве точности $\varepsilon$ были взяты следующие значения: $\{10^{-2}, 10^{-4}, 10^{-6}\}$.

\section*{Метод поразрядного поиска}

\imgH{height=210mm}{radix-search}{Схема алгоритма поразрядного поиска}

\section*{Результаты вычислений}

\begin{table}[H]
	\caption{Результаты вычислений}
	\begin{tabular}{|l|l|l|l|l|}
		\hline
		№ п/п & $\varepsilon$ & $N$  & $x^*$     & $f(x^*)$   \\ \hline
		$1$   & $10^{-2}$     & $19$ & $0,48046875$ & $-1,48046875$ \\ \hline
		$2$   & $10^{-4}$     & $36$ & $0,48242188$ & $-1,47389328$ \\ \hline
		$3$   & $10^{-6}$     & $50$ & $0,48241806$ & $-1,47389328$ \\ \hline
	\end{tabular}
\end{table}

\section*{Текст программы}

\lstinputlisting[
	caption={\code{lab01.m}},
	label={lst:lab01m},
]{../lab01.m}

\end{document}
