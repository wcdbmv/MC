\documentclass{bmstu-gost-7-32}

\begin{document}

\makereporttitle
	{Информатика, искусственный интеллект и системы управления} % Название факультета
	{Программное обеспечение ЭВМ и информационные технологии} % Название кафедры
	{лабораторной работе №~4} % Название работы (в дат. падеже)
	{Методы вычислений} % Название курса (необязательный аргумент)
	{Метод Ньютона} % Тема работы
	{6} % Номер варианта (необязательный аргумент)
	{ИУ7-23М} % Номер группы
	{Керимов~А.~Ш.} % ФИО студента
	{Власов~П.~А.} % ФИО преподавателя

\section*{Постановка задачи}

Решить одномерную задачу оптимизации вида
\begin{equation}
	\begin{cases}
		f(x) \to \min, \\
		x \in [a, b],
	\end{cases}
\end{equation}
методом Ньютона с заданной точностью $\varepsilon$.

\section*{Входные данные}

Заданная функция:
\begin{equation}
	f(x) = \ch{\left(\frac{3x^3 + 2x^2 - 4x + 5}{3}\right)} + \th{\left(\frac{x^3 - 3\sqrt2x - 2}{2x + \sqrt2}\right)} - 2,5.
\end{equation}

Поиск точки минимума производится на отрезке $[0, 1]$.
При построении таблицы результатов в качестве точности $\varepsilon$ были взяты следующие значения: $\{10^{-2}, 10^{-4}, 10^{-6}\}$.

\section*{Метод Ньютона}

Схема метода Ньютона представлена на рисунке \ref{img:newton}.

\imgH{width=0.9\linewidth}{newton}{Схема метода Ньютона}

\section*{Результаты вычислений}

\begin{table}[H]
	\caption{Результаты вычислений методом Ньютона}
	\begin{tabular}{|l|l|l|l|l|}
		\hline
		№ п/п & $\varepsilon$ & $N$  & $x^*$          & $f(x^*)$        \\ \hline
		$1$   & $10^{-2}$     & $11$ & $0,4774047325$ & $-1,4738017976$ \\ \hline
		$2$   & $10^{-4}$     & $14$ & $0,4823683107$ & $-1,4738932752$ \\ \hline
		$3$   & $10^{-6}$     & $17$ & $0,4824178114$ & $-1,4738932844$ \\ \hline
	\end{tabular}
\end{table}

\begin{table}[H]
	\caption{Сводная таблица для сравнения методов, $\varepsilon = 10^{-6}$}
	\begin{tabular}{|l|l|l|l|l|}
		\hline
		№ п/п & Метод               & $N$  & $x^*$          & $f(x^*)$        \\ \hline
		$1$   & поразрядного поиска & $50$ & $0,4824180651$ & $-1,4738932844$ \\ \hline
		$2$   & золотого сечения    & $31$ & $0,4824184749$ & $-1,4738932844$ \\ \hline
		$3$   & парабол             & $15$ & $0,4824178751$ & $-1,4738932844$ \\ \hline
		$4$   & Ньютона             & $17$ & $0,4824178114$ & $-1,4738932844$ \\ \hline
		$5$   & Функция fminbnd     & $10$ & $0,4824181903$ & $-1,4738932844$ \\ \hline
	\end{tabular}
\end{table}

\section*{Текст программы}

\lstinputlisting[
	caption={\code{lab04.m}},
	label={lst:lab04m},
]{../lab04.m}

\end{document}
