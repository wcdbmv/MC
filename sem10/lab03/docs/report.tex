\documentclass{bmstu-gost-7-32}

\begin{document}

\makereporttitle
	{Информатика, искусственный интеллект и системы управления} % Название факультета
	{Программное обеспечение ЭВМ и информационные технологии} % Название кафедры
	{лабораторной работе №~3} % Название работы (в дат. падеже)
	{Методы вычислений} % Название курса (необязательный аргумент)
	{Метод парабол} % Тема работы
	{6} % Номер варианта (необязательный аргумент)
	{ИУ7-23М} % Номер группы
	{Керимов~А.~Ш.} % ФИО студента
	{Власов~П.~А.} % ФИО преподавателя

\section*{Постановка задачи}

Решить одномерную задачу оптимизации вида
\begin{equation}
	\begin{cases}
		f(x) \to \min, \\
		x \in [a, b],
	\end{cases}
\end{equation}
методом парабол с заданной точностью $\varepsilon$.

\section*{Входные данные}

Заданная функция:
\begin{equation}
	f(x) = \ch{\left(\frac{3x^3 + 2x^2 - 4x + 5}{3}\right)} + \th{\left(\frac{x^3 - 3\sqrt2x - 2}{2x + \sqrt2}\right)} - 2,5.
\end{equation}

Поиск точки минимума производится на отрезке $[0, 1]$.
При построении таблицы результатов в качестве точности $\varepsilon$ были взяты следующие значения: $\{10^{-2}, 10^{-4}, 10^{-6}\}$.

\section*{Метод парабол}

Схема метода парабол представлена на рисунке \ref{img:successive-parabolic-interpolation}.

Условия выбора точек $x_1, x_2, x_3 \in [a, b)$:
\begin{enumerate}
	\item $x_1 < x_2 < x_3$,
	\item $f(x_1) \geqslant f(x_2) \leqslant f(x_3)$ принимает по крайней мере одно неравенство строгое.
\end{enumerate}

\imgH{height=215mm}{successive-parabolic-interpolation}{Схема метода парабол}

Замечания:
\begin{enumerate}
	\item В качестве критерия окончания вычислений используется условие $|\overline x - \overline x'| < \varepsilon$, означающее близость друг к другу двух последовательных приближений точки $x^*$ .
	Вообще говоря, выполнение этого условия не гарантирует близость этих точек к $x^*$.
	Однако на практике такое условие удовлетворительно работает.
	Дополнительно точность текущего приближения можно оценивать (если получится) с использованием длины отрезка $[x_1 , x_3]$.

	\item О выборе точек $x_1, x_2, x_3$
	\begin{enumerate}
		\item Можно выполнить несколько итераций метода золотого сечения до тех пор, пока пробные точки этого метода и одна из граничных точек текущего отрезка не будут удовлетворять условиям 1 и 2.

		\item На второй и последующих итерациях на отрезке $[x_1 , x_3]$ рассматриваются две пробные точки $x_2$ и $\overline x$, для которых используется метод исключения отрезков.
		В новом отрезке $[x_1', x_3']$ в качестве $x_2'$ выбирается та точка из $x_2$ и $\overline x$, которая оказалась внутри.
	\end{enumerate}
	\item На каждой итерации метода парабол, кроме первой, вычисляется только одно значение целевой функции $\overline f$.
\end{enumerate}

\section*{Результаты вычислений}

\begin{table}[H]
	\caption{Результаты вычислений}
	\begin{tabular}{|l|l|l|l|l|}
		\hline
		№ п/п & $\varepsilon$ & $N$  & $x^*$          & $f(x^*)$        \\ \hline
		$1$   & $10^{-2}$     & $9$  & $0,4773394983$ & $-1,4737994041$ \\ \hline
		$2$   & $10^{-4}$     & $13$ & $0,4824088798$ & $-1,4738932840$ \\ \hline
		$3$   & $10^{-6}$     & $15$ & $0,4824178751$ & $-1,4738932844$ \\ \hline
	\end{tabular}
\end{table}

\section*{Текст программы}

\lstinputlisting[
	caption={\code{lab03.m}},
	label={lst:lab03m},
]{../lab03.m}

\end{document}
