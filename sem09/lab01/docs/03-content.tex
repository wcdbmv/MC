%% ГОСТ 7.32-2017
%% 5.4 Содержание
%%
%% 5.4.1 Содержание включает введение, наименование всех разделов и подразделов, пунктов (если они имеют наименование), заключение, список использованных источников и наименования приложений с указанием номеров страниц, с которых начинаются эти элементы отчёта о НИР.
%%
%% В элементе "СОДЕРЖАНИЕ" приводят наименования структурных элементов работы, порядковые номера и заголовки разделов, подразделов (при необходимости — пунктов) основной части работы, обозначения и заголовки её приложений (при наличии приложений).
%% После заголовка каждого элемента ставят отточие и приводят номер страницы работы, на которой начинается данный структурный элемент.
%%
%% Обозначения подразделов приводят после абзацного отступа, равного двум знакам, относительно обозначения разделов.
%% Обозначения пунктов приводят после абзацного отступа, равного четырём знакам относительно обозначения разделов.
%%
%% При необходимости продолжение записи заголовка раздела, подраздела или пункта на второй (последующей) строке выполняют, начиная от уровня начала этого заголовка на первой строке, а продолжение записи заголовка приложения — от уровня записи обозначения этого приложения.
%%
%% 5.4.2 При составлении отчёта, состоящего из двух и более книг, в каждой из них должно быть приведено своё содержание.
%% При этом в первой книге помещают содержание всего отчёта с указанием номеров книг, в последующих — только содержание соответствующей книги.
%% Допускается в первой книге вместо содержания последующих книг указывать только их наименования.
%%
%% 5.4.3 Для отчёта о НИР объёмом не более 10 страниц содержание допускается не составлять.
%%
%% 5.4.4 Содержание следует оформлять в соответствии с 6.13.

\tableofcontents
