%% ГОСТ 7.32-2017
%% 5.1 Титульный лист
%%
%% 5.1.1 Титульный лист является первой страницей отчёта о НИР и служит источником информации, необходимой для обработки и поиска отчёта в информационной среде.
%%
%% 5.1.2 На титульном листе приводят следующие сведения:
%%     а) наименование министерства (ведомства) или другого структурного образования, в систему которого входит организация-исполнитель;
%%     б) наименование (полное и сокращённое) организации — исполнителя НИР;
%%     в) индекс Универсальной десятичной классификации (УДК) по ГОСТ 7.90;
%%     г) номера, идентифицирующие отчёт:
%%         1) регистрационный номер НИР [1] (присваивает национальный орган научно-технической информации каждой страны при открытии темы НИР);
%%            [1] В Российской Федерации регистрационный номер ЕГИСУ НИОКТР (Единая государственная информационная система учёта результатов научно-исследовательских, опытно-конструкторских и технологических работ гражданского назначения) присваивает ЦИТиС, который осуществляет учёт данных о научных исследованиях и разработках по всем областям.
%%         2) регистрационный номер отчёта [2] (присваивает национальный орган научно-технической информации каждой страны при предоставлении отчётной документации);
%%            [2] В Российской Федерации регистрационный номер ИКРБС (Информационная карта реферативно-библиографических сведений) присваивает ЦИТиС, который осуществляет формирование и поддержку национального библиотечно-информационного фонда РФ в части открытых неопубликованных источников научной и технической информации — отчётов о НИР и т.д.
%%     д) грифы согласования и утверждения отчёта, включая подпись руководителя организации с расшифровкой, печать организации и даты согласования и утверждения отчёта (дату указывают в интервале выполнения работы — для промежуточных отчётов и дату окончания — для заключительных отчётов);
%%     е) вид документа (отчёт о НИР);
%%     ж) наименование НИР;
%%     и) наименование отчёта;
%%     к) вид отчёта (заключительный, промежуточный);
%%     л) номер (шифр) научно-технической программы, темы;
%%     м) номер книги отчёта (при наличии нескольких книг отчёта);
%%     н) должность, учёную степень, учёное звание, подпись, инициалы и фамилию научного руководителя/руководителей НИР [3];
%%        [3] Для учреждений образования дополнительно аналогично вносятся подписи декана, заведующего кафедрой и других должностных лиц на усмотрение учреждений.
%%     п) место и год составления отчета.
%%
%% 5.1.3 Если отчёт о НИР состоит из двух и более книг, каждая книга должна иметь свой титульный лист, соответствующий титульному листу первой книги и содержащий сведения, относящиеся к данной книге.
%%
%% 5.1.4 Титульный лист следует оформлять в соответствии с 6.10.
%% Примеры оформления титульных листов отчёта о НИР приведены в приложении А.


\begin{titlepage}
	\centering

	\begin{wrapfigure}[7]{l}{0.14\linewidth}
		\vspace{5mm}
		\hspace{-8mm}
		\includegraphics[width=0.89\linewidth]{inc/img/bmstu-logo}
	\end{wrapfigure}

	{\singlespacing\footnotesize\bfseries
		Министерство науки и высшего образования Российской Федерации \\
		Федеральное государственное бюджетное образовательное учреждение \\
		высшего образования \\
		«Московский государственный технический университет \\
		имени Н.~Э.~Баумана \\
		(национальный исследовательский университет)» \\
		(МГТУ им. Н.~Э.~Баумана) \\
	}

	\vspace{-2.2mm}
	\vhrulefill{0.9mm} \\
	\vspace{-7mm}
	\vhrulefill{0.2mm} \\
	\vspace{2mm}

	{\doublespacing\small\raggedright
		ФАКУЛЬТЕТ \hspace{28mm} «Информатика и системы управления» \\
		КАФЕДРА \hspace{9mm} «Программное обеспечение ЭВМ и информационные технологии» \\
	}

	\vspace{20mm}

	{\large\bfseries ОТЧЁТ} \\
	По лабораторной работе № 1 \\
	По дисциплине: «Методы вычислений» \\
	На тему: {\itshape «Венгерский метод решения задачи о назначениях»} \\
	Вариант 6

	\vfill

	\begin{tabular}{p{0.52\textwidth} c c c}
		Студент \underline{ИУ7-13М}            & \underline{\hspace{35mm}}     & \underline{Керимов~А.~Ш.}     \\ [-0.6em]
		{\hspace{23.5mm} \scriptsize (Группа)} & {\scriptsize (Подпись, дата)} & {\scriptsize (Фамилия~И.~О.)} \\
		Преподаватель                          & \underline{\hspace{35mm}}     & \underline{Власов~П.~А.}      \\ [-0.6em]
		                                       & {\scriptsize (Подпись, дата)} & {\scriptsize (Фамилия~И.~О.)} \\
	\end{tabular}

	\vspace{15mm}

	Москва {\the\year}
\end{titlepage}

\setcounter{page}{2}
