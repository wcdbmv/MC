%% МЕЖГОСУДАРСТВЕННЫЙ СТАНДАРТ ГОСТ 7.32-2017
%%
%% Система стандартов по информации, библиотечному и издательскому делу
%% ОТЧЁТ О НАУЧНО-ИССЛЕДОВАТЕЛЬСКОЙ РАБОТЕ
%% Структура и правила оформления
%%
%% System of standards on information, librarianship and publishing. The research report. Structure and rules of presentation


%% ГОСТ 7.32-2017
%% 1 Область применения
%%
%% Настоящий стандарт устанавливает общие требования к структуре и правилам оформления отчётов о научно-исследовательских, проектно-конструкторских, конструкторско-технологических и проектно-технологических работах (далее — отчётов о НИР), а также для тех случаев, когда единая процедура оформления будет содействовать обмену информацией, совершенствуя обработку отчёта в информационной системе.
%%
%% Настоящий стандарт распространяется на отчёты о фундаментальных, поисковых и прикладных научно-исследовательских работах по всем областям науки и техники, выполняемых научно-исследовательскими, проектными, конструкторскими организациями, высшими учебными заведениями, научно-производственными объединениями и другими организациями независимо от их организационно-правовой формы.
%%
%% Положения настоящего стандарта могут быть использованы при подготовке отчёта о НИР в других областях научной деятельности.


%% ГОСТ 7.32-2017
%% 3 Общие положения
%%
%% 3.1 Отчёт о НИР — документ, который содержит систематизированные данные о научно-исследовательской работе, описывает состояние научно-технической проблемы, процесс, результаты научно-технического исследования.
%%
%% 3.2 По результатам выполнения НИР составляется заключительный отчёт о работе в целом.
%% Кроме того, по отдельным этапам НИР могут быть составлены промежуточные отчёты в соответствии с настоящим стандартом и ГОСТ 15.101, что отражается в техническом задании на НИР и в календарном плане выполнения НИР.
%%
%% 3.3 Заключительные отчеты обязательно направляются организацией — исполнителем НИР в соответствующий орган научно-технической информации в соответствии с порядком, установленным законодательством страны.
%%
%% 3.4 Ответственность за достоверность данных, содержащихся в отчёте о НИР, и за соответствие его требованиям настоящего стандарта несёт организация — исполнитель НИР.
%%
%% 3.5 Отчёт о НИР подлежит обязательному нормоконтролю в организации-исполнителе.
%% При проведении нормоконтроля рекомендуется руководствоваться настоящим стандартом.
%%
%% 3.6 Отчёт оформляется на национальном языке каждой страны или на русском языке, который является официальным языком Межгосударственного совета по стандартизации, метрологии и сертификации.
%% Допускается в отчётах по общественным наукам использовать национальный и русский языки.


%% ГОСТ 7.32-2017
%% 6.1 Общие требования
%%
%% 6.1.1 Изложение текста и оформление отчёта выполняют в соответствии с требованиями настоящего стандарта.
%% Страницы текста отчёта о НИР и включенные в отчёт иллюстрации и таблицы должны соответствовать формату А4 по ГОСТ 9327.
%% Допускается применение формата А3 при наличии большого количества таблиц и иллюстраций данного формата.

\documentclass[14pt, a4paper]{extreport}


\usepackage[T2A]{fontenc}
\usepackage[utf8]{inputenc}
\usepackage[main=russian, english]{babel}

\usepackage{microtype}
\sloppy


%% ГОСТ 7.32-2017
%% 6.1 Общие требования
%%
%% 6.1.1 Изложение текста и оформление отчёта выполняют в соответствии с требованиями настоящего стандарта.
%% Страницы текста отчёта о НИР и включённые в отчёт иллюстрации и таблицы должны соответствовать формату А4 по ГОСТ 9327.
%% Допускается применение формата А3 при наличии большого количества таблиц и иллюстраций данного формата.
%%
%% Отчёт о НИР должен быть выполнен любым печатным способом на одной стороне листа белой бумаги формата А4 через полтора интервала.

\usepackage[onehalfspacing]{setspace}

%% Допускается при подготовке заключительного отчёта о НИР печатать через один интервал, если отчёт имеет значительный объём (500 и более страниц).
%% Цвет шрифта должен быть чёрным, размер шрифта — не менее 12 пт.
%% Рекомендуемый тип шрифта для основного текста отчёта — Times New Roman.

\usepackage{paratype}
\renewcommand{\rmdefault}{PTSerif-TLF}
\renewcommand{\ttdefault}{PTMono-TLF}

%% Полужирный шрифт применяют только для заголовков разделов и подразделов, заголовков структурных элементов.
%% Использование курсива допускается для обозначения объектов (биология, геология, медицина, нанотехнологии, генная инженерия и др.) и написания терминов (например, in vivo, in vitro) и иных объектов и терминов на латыни.
%%
%% Для акцентирования внимания может применяться выделение текста с помощью шрифта иного начертания, чем шрифт основного текста, но того же кегля и гарнитуры.
%% Разрешается для написания определённых терминов, формул, теорем применять шрифты разной гарнитуры.
%%
%% Текст отчёта следует печатать, соблюдая следующие размеры полей: левое — 30 мм, правое — 15 мм, верхнее и нижнее — 20 мм.

\newcommand{\PageLeftMargin}{30mm}
\newcommand{\PageRightMargin}{15mm}
\newcommand{\PageTopMargin}{20mm}
\newcommand{\PageBottomMargin}{20mm}

\usepackage[
	left=\PageLeftMargin,
	right=\PageRightMargin,
	top=\PageTopMargin,
	bottom=\PageBottomMargin,
]{geometry}

%% Абзацный отступ должен быть одинаковым по всему тексту отчёта и равен 1,25 см.

\usepackage{indentfirst}
\setlength{\parindent}{1.25cm}

\setlength{\parskip}{1ex}

%%
%% 6.1.2 Вне зависимости от способа выполнения отчёта качество напечатанного текста и оформления иллюстраций, таблиц, распечаток программ должно удовлетворять требованию их чёткого воспроизведения.
%%
%% 6.1.3 При выполнении отчёта о НИР необходимо соблюдать равномерную плотность и чёткость изображения по всему отчёту.
%% Все линии, буквы, цифры и знаки должны иметь одинаковую контрастность по всему тексту отчёта.
%%
%% 6.1.4 Фамилии, наименования учреждений, организаций, фирм, наименования изделий и другие имена собственные в отчёте приводят на языке оригинала.
%% Допускается транслитерировать имена собственные и приводить наименования организаций в переводе на язык отчёта с добавлением (при первом упоминании) оригинального названия по ГОСТ 7.79.
%%
%% 6.1.5 Сокращения слов и словосочетаний на русском, белорусском [1] и иностранных европейских языках оформляют в соответствии с требованиями ГОСТ 7.11, ГОСТ 7.12.
%%   [1] Для Республики Беларусь применим СТБ 7.12.
%%
%% 6.2 Построение отчёта
%%
%% 6.2.1 Наименования структурных элементов отчёта: "СПИСОК ИСПОЛНИТЕЛЕЙ", "РЕФЕРАТ", "СОДЕРЖАНИЕ", "ТЕРМИНЫ И ОПРЕДЕЛЕНИЯ", "ПЕРЕЧЕНЬ СОКРАЩЕНИЙ И ОБОЗНАЧЕНИЙ", "ВВЕДЕНИЕ", "ЗАКЛЮЧЕНИЕ", "СПИСОК ИСПОЛЬЗОВАННЫХ ИСТОЧНИКОВ", "ПРИЛОЖЕНИЕ" служат заголовками структурных элементов отчёта.

\AtBeginDocument{\renewcommand\contentsname{Содержание}}

%%
%% Заголовки структурных элементов следует располагать в середине строки без точки в конце, прописными буквами, не подчёркивая.
%% Каждый структурный элемент и каждый раздел основной части отчёта начинают с новой страницы.

\usepackage{titlesec}

\newcommand{\ChapterLeftMargin}{}
\newcommand{\ChapterFormat}{}
\newcommand{\ChapterSep}{}
\newcommand{\ChapterBeforeCode}{}

\newcommand{\SetMainPartChapterSettings}{%
	\renewcommand{\ChapterLeftMargin}{\parindent}%
	\renewcommand{\ChapterFormat}{\bfseries}%
	\renewcommand{\ChapterSep}{1em}%
	\renewcommand{\ChapterBeforeCode}{}%
}

\newcommand{\SetStructuralElementChapterSettings}{%
	\renewcommand{\ChapterLeftMargin}{0pt}%
	\renewcommand{\ChapterFormat}{\filcenter\bfseries}%
	\renewcommand{\ChapterSep}{}%
	\renewcommand{\ChapterBeforeCode}{\MakeUppercase}%
}

\SetStructuralElementChapterSettings

\titlespacing*{\chapter}{\ChapterLeftMargin}{-30pt}{8pt}
\titleformat{\chapter}[block]{\ChapterFormat}{\thechapter}{\ChapterSep}{\ChapterBeforeCode}{}

\newcommand{\StructuralElement}[1]{%
	\chapter*{#1}%
	\addcontentsline{toc}{chapter}{\MakeUppercase{#1}}%
}

\newenvironment{MainPart}%
	{\SetMainPartChapterSettings}%
	{\SetStructuralElementChapterSettings}

%%
%% 6.2.2 Основную часть отчёта следует делить на разделы, подразделы и пункты.
%% Пункты при необходимости могут делиться на подпункты.
%% Разделы и подразделы отчёта должны иметь заголовки.
%% Пункты и подпункты, как правило, заголовков не имеют.
%%
%% 6.2.3 Заголовки разделов и подразделов основной части отчёта следует начинать с абзацного отступа и размещать после порядкового номера, печатать с прописной буквы, полужирным шрифтом, не подчёркивать, без точки в конце.
%% Пункты и подпункты могут иметь только порядковый номер без заголовка, начинающийся с абзацного отступа.

\titlespacing*{\section}{\parindent}{*4}{*2}
\titlespacing*{\subsection}{\parindent}{*4}{*4}
\titlespacing*{\subsubsection}{\parindent}{*4}{*4}
\titlespacing*{\paragraph}{\parindent}{*0}{*1}
\titleformat{\section}{\bfseries}{\thesection}{1em}{}{}{}
\titleformat{\subsection}{\bfseries}{\thesubsection}{1em}{}{}{}
\titleformat{\subsubsection}{\bfseries}{\thesubsubsection}{1em}{}{}{}

%%
%% 6.2.4 Если заголовок включает несколько предложений, их разделяют точками.
%% Переносы слов в заголовках не допускаются.
%%
%% 6.3 Нумерация страниц отчёта
%%
%% 6.3.1 Страницы отчёта следует нумеровать арабскими цифрами, соблюдая сквозную нумерацию по всему тексту отчёта, включая приложения.
%% Номер страницы проставляется в центре нижней части страницы без точки.
%% Приложения, которые приведены в отчёте о НИР и имеющие собственную нумерацию, допускается не перенумеровать.
%%
%% 6.3.2 Титульный лист включают в общую нумерацию страниц отчёта.
%% Номер страницы на титульном листе не проставляют.
%%
%% 6.3.3 Иллюстрации и таблицы, расположенные на отдельных листах, включают в общую нумерацию страниц отчёта.
%% Иллюстрации и таблицы на листе формата А3 учитывают как одну страницу.
%%
%% 6.4 Нумерация разделов, подразделов, пунктов, подпунктов и книг отчёта
%%
%% 6.4.1 Разделы должны иметь порядковые номера в пределах всего отчёта, обозначенные арабскими цифрами без точки и расположенные с абзацного отступа.
%% Подразделы должны иметь нумерацию в пределах каждого раздела.
%% Номер подраздела состоит из номеров раздела и подраздела, разделённых точкой.
%% В конце номера подраздела точка не ставится.
%% Разделы, как и подразделы, могут состоять из одного или нескольких пунктов.
%%
%% 6.4.2 Если отчёт не имеет подразделов, то нумерация пунктов в нем должна быть в пределах каждого раздела и номер пункта должен состоять из номеров раздела и пункта, разделённых точкой.
%% В конце номера пункта точка не ставится.
%%
%% Если отчёт имеет подразделы, то нумерация пунктов должна быть в пределах подраздела и номер пункта должен состоять из номеров раздела, подраздела и пункта, разделённых точками.
%%
%% 6.4.3 Если раздел или подраздел состоит из одного пункта, то пункт не нумеруется.
%%
%% 6.4.4 Если текст отчёта подразделяется только на пункты, они нумеруются порядковыми номерами в пределах отчёта.
%%
%% 6.4.5 Пункты при необходимости могут быть разбиты на подпункты, которые должны иметь порядковую нумерацию в пределах каждого пункта: 4.2.1.1, 4.2.1.2, 4.2.1.3 и т.д.

\setcounter{tocdepth}{3}
\setcounter{secnumdepth}{3}

%%
%% 6.4.6 Внутри пунктов или подпунктов могут быть приведены перечисления.
%% Перед каждым элементом перечисления следует ставить тире.

\renewcommand{\labelitemi}{—}
\renewcommand{\labelitemii}{—}

%% При необходимости ссылки в тексте отчёта на один из элементов перечисления вместо тире ставят строчные буквы русского алфавита со скобкой, начиная с буквы "а" (за исключением букв ё, з, й, о, ч, ъ, ы, ь).

\renewcommand{\labelenumi}{\asbuk{enumi})}
\renewcommand{\labelenumii}{\arabic{enumii})}
\usepackage{enumitem}

\makeatletter
	\AddEnumerateCounter{\asbuk}{\@asbuk}{ю)}
\makeatother
\setlist{nosep, leftmargin=\parindent}

%% Простые перечисления отделяются запятой, сложные — точкой с запятой.
%%
%% При наличии конкретного числа перечислений допускается перед каждым элементом перечисления ставить арабские цифры, после которых ставится скобка.
%%
%% Перечисления приводятся с абзацного отступа в столбик.
%%
%% 6.4.7 Заголовки должны чётко и кратко отражать содержание разделов, подразделов.
%% Если заголовок состоит из двух предложений, их разделяют точкой.
%%
%% 6.4.8 Если отчёт состоит из двух и более книг, каждая книга должна иметь свой порядковый номер.
%% Номер каждой книги следует проставлять арабскими цифрами на титульном листе под указанием вида отчёта: "Книга 2".
%%
%% 6.5 Иллюстрации

\usepackage{graphicx}
\usepackage{float}

\usepackage[tableposition=top, singlelinecheck=false]{caption}
\usepackage{subcaption}

\DeclareCaptionLabelFormat{gostfigure}{Рисунок #2}
\DeclareCaptionLabelFormat{gosttable}{Таблица #2}
\DeclareCaptionLabelSeparator{gost}{~—~}
\captionsetup{labelsep=gost}
\captionsetup*[figure]{labelformat=gostfigure}
\captionsetup*[table]{labelformat=gosttable}
\renewcommand{\thesubfigure}{\asbuk{subfigure}}

%%
%% 6.5.1 Иллюстрации (чертежи, графики, схемы, компьютерные распечатки, диаграммы, фотоснимки) следует располагать в отчёте непосредственно после текста отчёта, где они упоминаются впервые, или на следующей странице (по возможности ближе к соответствующим частям текста отчёта).
%% На все иллюстрации в отчёте должны быть даны ссылки.
%% При ссылке необходимо писать слово "рисунок" и его номер, например: "в соответствии с рисунком 2" и т.д.

\newcommand{\imght}[3]{%
	\begin{figure}[ht]%
		\center{\includegraphics[#1]{inc/img/#2}}%
		\captionsetup{justification=centering}%
		\caption{#3}%
		\label{img:#2}%
	\end{figure}%
}

\newcommand{\imgH}[3]{%
	\begin{figure}[H]%
		\center{\includegraphics[#1]{inc/img/#2}}%
		\captionsetup{justification=centering}%
		\caption{#3}%
		\label{img:#2}%
	\end{figure}%
}

\usepackage{pgfplots}

%%
%% 6.5.2 Чертежи, графики, диаграммы, схемы, помещаемые в отчёте, должны соответствовать требованиям стандартов Единой системы конструкторской документации (ЕСКД).
%%
%% 6.5.3 Количество иллюстраций должно быть достаточным для пояснения излагаемого текста отчёта.
%% Не рекомендуется в отчёте о НИР приводить объёмные рисунки.
%%
%% 6.5.4 Иллюстрации, за исключением иллюстраций, приведённых в приложениях, следует нумеровать арабскими цифрами сквозной нумерацией.
%% Если рисунок один, то он обозначается: Рисунок 1.
%%   Пример — Рисунок 1 — Схема прибора
%%
%% 6.5.5 Иллюстрации каждого приложения обозначают отдельной нумерацией арабскими цифрами с добавлением перед цифрой обозначения приложения: Рисунок А.3.
%%
%% 6.5.6 Допускается нумеровать иллюстрации в пределах раздела отчёта.
%% В этом случае номер иллюстрации состоит из номера раздела и порядкового номера иллюстрации, разделённых точкой: Рисунок 2.1.
%%
%% 6.5.7 Иллюстрации при необходимости могут иметь наименование и пояснительные данные (подрисуночный текст).
%% Слово "Рисунок", его номер и через тире наименование помещают после пояснительных данных и располагают в центре под рисунком без точки в конце.
%%   Пример — Рисунок 2 — Оформление таблицы
%%
%% 6.5.8 Если наименование рисунка состоит из нескольких строк, то его следует записывать через один межстрочный интервал.
%% Наименование рисунка приводят с прописной буквы без точки в конце.
%% Перенос слов в наименовании графического материала не допускается.
%%
%% 6.6 Таблицы

\renewcommand{\arraystretch}{1.3}

\newcommand{\TableHeader}[2]{%
	\parbox{#1}{%
		\vspace{.5\baselineskip}%
		\centering{\textbf{#2}}%
		\vspace{.5\baselineskip}%
	}%
}

%%
%% 6.6.1 Цифровой материал должен оформляться в виде таблиц.
%% Таблицы применяют для наглядности и удобства сравнения показателей.
%%
%% 6.6.2 Таблицу следует располагать непосредственно после текста, в котором она упоминается впервые, или на следующей странице.
%%
%% На все таблицы в отчёте должны быть ссылки.
%% При ссылке следует печатать слово "таблица" с указанием её номера.
%%
%% 6.6.3 Наименование таблицы, при ее* наличии, должно отражать её содержание, быть точным, кратким.
%% Наименование следует помещать над таблицей слева, без абзацного отступа в следующем формате: Таблица Номер таблицы — Наименование таблицы.
%% Наименование таблицы приводят с прописной буквы без точки в конце.
%%   * Вероятно ошибка оригинала. Следует читать "его". - Примечание изготовителя базы данных.
%%
%% Если наименование таблицы занимает две строки и более, то его следует записывать через один межстрочный интервал.
%%
%% Таблицу с большим количеством строк допускается переносить на другую страницу.
%% При переносе части таблицы на другую страницу слово "Таблица", её номер и наименование указывают один раз слева над первой частью таблицы, а над другими частями также слева пишут слова "Продолжение таблицы" и указывают номер таблицы.
%%
%% При делении таблицы на части допускается её головку или боковик заменять соответственно номерами граф и строк.
%% При этом нумеруют арабскими цифрами графы и (или) строки первой части таблицы.
%% Таблица оформляется в соответствии с рисунком 1.
%%
%% 6.6.4 Таблицы, за исключением таблиц приложений, следует нумеровать арабскими цифрами сквозной нумерацией.
%%
%% Таблицы каждого приложения обозначаются отдельной нумерацией арабскими цифрами с добавлением перед цифрой обозначения приложения.
%% Если в отчёте одна таблица, она должна быть обозначена "Таблица 1" или "Таблица А.1" (если она приведена в приложении А).
%%
%% Допускается нумеровать таблицы в пределах раздела при большом объёме отчёта.
%% В этом случае номер таблицы состоит из номера раздела и порядкового номера таблицы, разделённых точкой: Таблица 2.3.
%%
%% 6.6.5 Заголовки граф и строк таблицы следует печатать с прописной буквы, а подзаголовки граф — со строчной буквы, если они составляют одно предложение с заголовком, или с прописной буквы, если они имеют самостоятельное значение.
%% В конце заголовков и подзаголовков таблиц точки не ставятся.
%% Названия заголовков и подзаголовков таблиц указывают в единственном числе.
%%
%% 6.6.6 Таблицы слева, справа, сверху и снизу ограничивают линиями.
%% Разделять заголовки и подзаголовки боковика и граф диагональными линиями не допускается.
%% Заголовки граф выравнивают по центру, а заголовки строк — по левому краю.
%%
%% Горизонтальные и вертикальные линии, разграничивающие строки таблицы, допускается не проводить, если их отсутствие не затрудняет пользование таблицей.
%%
%% 6.6.7 Текст, повторяющийся в строках одной и той же графы и состоящий из одиночных слов, заменяют кавычками.
%% Ставить кавычки вместо повторяющихся цифр, буквенно-цифровых обозначений, знаков и символов не допускается.
%%
%% Если текст повторяется, то при первом повторении его заменяют словами "то же", а далее кавычками.
%%
%% В таблице допускается применять размер шрифта меньше, чем в тексте отчёта.
%%
%% 6.7 Примечания и сноски

\renewcommand*{\thefootnote}{\arabic{footnote})}
\renewcommand{\footnoterule}{%
	\kern -3pt%
	\hrule width 40mm height .4pt%
	\kern 2.6pt%
}

%%
%% 6.7.1 Примечания приводят в отчёте, если необходимы пояснения или справочные данные к содержанию текста, таблиц или графического материала.
%%
%% 6.7.2 Слово "Примечание" следует печатать с прописной буквы с абзацного отступа, не подчёркивая.
%%
%% 6.7.3 Примечания следует помещать непосредственно после текстового, графического материала или таблицы, к которым относятся эти примечания.
%% Если примечание одно, то после слова "Примечание" ставится тире и текст примечания печатают с прописной буквы.
%% Одно примечание не нумеруется. Несколько примечаний нумеруют по порядку арабскими цифрами без точки.
%%
%% 6.7.4 При необходимости дополнительного пояснения в отчёте допускается использовать примечание, оформленное в виде сноски.
%% Знак сноски ставят без пробела непосредственно после того слова, числа, символа, предложения, к которому даётся пояснение.
%% Знак сноски указывается надстрочно арабскими цифрами.
%% Допускается вместо цифр использовать знак звёздочка - *.
%%
%% Сноску располагают с абзацного отступа в конце страницы, на которой приведено поясняемое слово (словосочетание или данные).
%% Сноску отделяют от текста короткой сплошной тонкой горизонтальной линией с левой стороны страницы.
%%
%% 6.8 Формулы и уравнения

\usepackage{amsmath}
\usepackage{amssymb}
\usepackage{icomma}
\usepackage{mathtools}

\DeclareMathOperator{\sign}{sign}

%%
%% 6.8.1 Уравнения и формулы следует выделять из текста в отдельную строку.
%% Выше и ниже каждой формулы или уравнения должно быть оставлено не менее одной свободной строки.
%% Если уравнение не умещается в одну строку, оно должно быть перенесено после знака равенства (=) или после знаков плюс (+), минус (-), умножения (х), деления (:) или других математических знаков.
%% На новой строке знак повторяется.
%% При переносе формулы на знаке, символизирующем операцию умножения, применяют знак "X".
%%
%% 6.8.2 Пояснение значений символов и числовых коэффициентов следует приводить непосредственно под формулой в той же последовательности, в которой они представлены в формуле.
%% Значение каждого символа и числового коэффициента необходимо приводить с новой строки.
%% Первую строку пояснения начинают со слова "где" без двоеточия с абзаца.
%%
%% 6.8.3 Формулы в отчёте следует располагать посередине строки и обозначать порядковой нумерацией в пределах всего отчёта арабскими цифрами в круглых скобках в крайнем правом положении на строке.
%% Одну формулу обозначают (1).
%%
%% 6.8.4 Ссылки в отчёте на порядковые номера формул приводятся в скобках: в формуле (1).
%%
%% 6.8.5 Формулы, помещаемые в приложениях, нумеруются арабскими цифрами в пределах каждого приложения с добавлением перед каждой цифрой обозначения приложения: (В.1).
%%
%% Допускается нумерация формул в пределах раздела.
%% В этом случае номер формулы состоит из номера раздела и порядкового номера формулы, разделённых точкой: (3.1).
%%
%% 6.9 Ссылки

\usepackage[unicode]{hyperref}
\hypersetup{hidelinks}

\usepackage{csquotes}
\usepackage[%
	backend=biber,
	bibencoding=utf8,
	language=auto,
	style=gost-numeric,
	sorting=none,
]{biblatex}
\addbibresource{91-references.bib}

% Режим доступа вместо URL в biblatex
\DeclareFieldFormat{url}{Режим доступа\addcolon\space\url{#1}}
% Дата обращения без сокращений в biblatex
\DeclareFieldFormat{urldate}{(дата обращения\addcolon\space\urldate{#1})}

%%
%% 6.9.1 В отчёте о НИР рекомендуется приводить ссылки на использованные источники.
%% При нумерации ссылок на документы, использованные при составлении отчёта, приводится сплошная нумерация для всего текста отчёта в целом или для отдельных разделов.
%% Порядковый номер ссылки (отсылки) приводят арабскими цифрами в квадратных скобках в конце текста ссылки.
%% Порядковый номер библиографического описания источника в списке использованных источников соответствует номеру ссылки.
%%
%% 6.9.2 Ссылаться следует на документ в целом или на его разделы и приложения.
%%
%% 6.9.3 При ссылках на стандарты и технические условия указывают их обозначение, при этом допускается не указывать год их утверждения при условии полного описания стандарта и технических условий в списке использованных источников в соответствии с ГОСТ 7.1.


%% for title-page
\usepackage{wrapfig}
%%
\makeatletter
	\def\vhrulefill#1{\leavevmode\leaders\hrule\@height#1\hfill\kern\z@}
\makeatother


%% for listings
\usepackage{algorithm}
\usepackage{algpseudocode}
%%
\renewcommand{\listalgorithmname}{Список алгоритмов}
\floatname{algorithm}{Алгоритм}
\algrenewcommand\algorithmicwhile{\textbf{До тех пор, пока}}
\algrenewcommand\algorithmicdo{\textbf{выполнять}}
\algrenewcommand\algorithmicrepeat{\textbf{Повторять}}
\algrenewcommand\algorithmicuntil{\textbf{Пока выполняется}}
\algrenewcommand\algorithmicend{\textbf{Конец}}
\algrenewcommand\algorithmicif{\textbf{Если}}
\algrenewcommand\algorithmicelse{\textbf{Иначе}}
\algrenewcommand\algorithmicthen{\textbf{тогда}}
\algrenewcommand\algorithmicfor{\textbf{Цикл}}
\algrenewcommand\algorithmicforall{\textbf{Для всех}}
\algrenewcommand\algorithmicfunction{\textbf{Функция}}
\algrenewcommand\algorithmicprocedure{\textbf{Процедура}}
\algrenewcommand\algorithmicloop{\textbf{Зациклить}}
\algrenewcommand\algorithmicrequire{\textbf{Условия:}}
\algrenewcommand\algorithmicensure{\textbf{Обеспечивающие условия:}}
\algrenewcommand\algorithmicreturn{\textbf{Возвратить}}
\algnewcommand{\algorithmicgoto}{\textbf{Перейти к}}
\algnewcommand{\Goto}[1]{\algorithmicgoto~\ref{#1} шагу алгоритма}
\algrenewtext{EndWhile}{\textbf{Конец цикла}}
\algrenewtext{EndLoop}{\textbf{Конец зацикливания}}
\algrenewtext{EndFor}{\textbf{Конец цикла}}
\algrenewtext{EndFunction}{\textbf{Конец функции}}
\algrenewtext{EndProcedure}{\textbf{Конец процедуры}}
\algrenewtext{EndIf}{\textbf{Конец условия}}
\algrenewtext{EndFor}{\textbf{Конец цикла}}
\algrenewtext{BeginAlgorithm}{\textbf{Начало алгоритма}}
\algrenewtext{EndAlgorithm}{\textbf{Конец алгоритма}}
\algrenewtext{BeginBlock}{\textbf{Начало блока. }}
\algrenewtext{EndBlock}{\textbf{Конец блока}}
\algrenewtext{ElsIf}{\textbf{иначе если }}
%%
\usepackage{matlab-prettifier}
\usepackage{listings}
\usepackage{listingsutf8}
\usepackage{xcolor}
\let\ph\mlplaceholder % shorter macro
\lstMakeShortInline"

\lstset{
	style              = Matlab-editor,
	basicstyle         = \scriptsize\ttfamily,
	breakatwhitespace  = true,
	breaklines         = true,
	frame              = single,
	inputencoding      = utf8/koi8-r,
	mlshowsectionrules = true,
	numbers            = left,
	numbersep          = 5pt,
	numberstyle        = \tiny\ttfamily\color{gray},
	tabsize            = 2,
}
%%
\newcommand{\code}[1]{\texttt{#1}}

%% reduce indent in contents
%% https://tex.stackexchange.com/questions/409569/change-indent-in-standard-table-of-content-not-tocloft
\usepackage{etoolbox}
\makeatletter
	% \patchcmd{<cmd>}{<search>}{<replace>}{<success>}{<failure>}
	\patchcmd{\l@section}{1.5em}{1em}{}{}
	\patchcmd{\l@subsection}{3.8em}{2em}{}{}
	\patchcmd{\l@subsubsection}{7.0em}{3em}{}{}
\makeatother



%% ГОСТ 7.32-2017
%% 4 Структурные элементы отчёта
%%
%% Структурными элементами отчёта о НИР являются:
%%     — титульный лист;
%%     — список исполнителей;
%%     — реферат;
%%     — содержание;
%%     ~ термины и определения;
%%     ~ перечень сокращений и обозначений;
%%     — введение;
%%     — основная часть отчёта о НИР;
%%     — заключение;
%%     ~ список использованных источников;
%%     ~ приложения.
%%
%% Обязательные структурные элементы выделены полужирным шрифтом.
%% Остальные структурные элементы включают в отчёт о НИР по усмотрению исполнителя НИР с учётом требований разделов 5 и 6.


%%%% Содержание отчета
%%%% 1. содержательная и математическая постановки задачи о назаначениях, а также исходные данные конкретного варианта;
%%%% 2. краткое описание венгерского метода(можно в «псевдокодах»);
%%%% 3. текст программы;
%%%% 4. результаты расчетов для задач из индивидуального варианта.

\begin{document}

%% ГОСТ 7.32-2017
%% 5.1 Титульный лист
%%
%% 5.1.1 Титульный лист является первой страницей отчёта о НИР и служит источником информации, необходимой для обработки и поиска отчёта в информационной среде.
%%
%% 5.1.2 На титульном листе приводят следующие сведения:
%%     а) наименование министерства (ведомства) или другого структурного образования, в систему которого входит организация-исполнитель;
%%     б) наименование (полное и сокращённое) организации — исполнителя НИР;
%%     в) индекс Универсальной десятичной классификации (УДК) по ГОСТ 7.90;
%%     г) номера, идентифицирующие отчёт:
%%         1) регистрационный номер НИР [1] (присваивает национальный орган научно-технической информации каждой страны при открытии темы НИР);
%%            [1] В Российской Федерации регистрационный номер ЕГИСУ НИОКТР (Единая государственная информационная система учёта результатов научно-исследовательских, опытно-конструкторских и технологических работ гражданского назначения) присваивает ЦИТиС, который осуществляет учёт данных о научных исследованиях и разработках по всем областям.
%%         2) регистрационный номер отчёта [2] (присваивает национальный орган научно-технической информации каждой страны при предоставлении отчётной документации);
%%            [2] В Российской Федерации регистрационный номер ИКРБС (Информационная карта реферативно-библиографических сведений) присваивает ЦИТиС, который осуществляет формирование и поддержку национального библиотечно-информационного фонда РФ в части открытых неопубликованных источников научной и технической информации — отчётов о НИР и т.д.
%%     д) грифы согласования и утверждения отчёта, включая подпись руководителя организации с расшифровкой, печать организации и даты согласования и утверждения отчёта (дату указывают в интервале выполнения работы — для промежуточных отчётов и дату окончания — для заключительных отчётов);
%%     е) вид документа (отчёт о НИР);
%%     ж) наименование НИР;
%%     и) наименование отчёта;
%%     к) вид отчёта (заключительный, промежуточный);
%%     л) номер (шифр) научно-технической программы, темы;
%%     м) номер книги отчёта (при наличии нескольких книг отчёта);
%%     н) должность, учёную степень, учёное звание, подпись, инициалы и фамилию научного руководителя/руководителей НИР [3];
%%        [3] Для учреждений образования дополнительно аналогично вносятся подписи декана, заведующего кафедрой и других должностных лиц на усмотрение учреждений.
%%     п) место и год составления отчета.
%%
%% 5.1.3 Если отчёт о НИР состоит из двух и более книг, каждая книга должна иметь свой титульный лист, соответствующий титульному листу первой книги и содержащий сведения, относящиеся к данной книге.
%%
%% 5.1.4 Титульный лист следует оформлять в соответствии с 6.10.
%% Примеры оформления титульных листов отчёта о НИР приведены в приложении А.


\begin{titlepage}
	\centering

	\begin{wrapfigure}[7]{l}{0.14\linewidth}
		\vspace{5mm}
		\hspace{-8mm}
		\includegraphics[width=0.89\linewidth]{inc/img/bmstu-logo}
	\end{wrapfigure}

	{\singlespacing\footnotesize\bfseries
		Министерство науки и высшего образования Российской Федерации \\
		Федеральное государственное бюджетное образовательное учреждение \\
		высшего образования \\
		«Московский государственный технический университет \\
		имени Н.~Э.~Баумана \\
		(национальный исследовательский университет)» \\
		(МГТУ им. Н.~Э.~Баумана) \\
	}

	\vspace{-2.2mm}
	\vhrulefill{0.9mm} \\
	\vspace{-7mm}
	\vhrulefill{0.2mm} \\
	\vspace{2mm}

	{\doublespacing\small\raggedright
		ФАКУЛЬТЕТ \hspace{28mm} «Информатика и системы управления» \\
		КАФЕДРА \hspace{9mm} «Программное обеспечение ЭВМ и информационные технологии» \\
	}

	\vspace{20mm}

	{\large\bfseries ОТЧЁТ} \\
	По лабораторной работе № 1 \\
	По дисциплине: «Методы вычислений» \\
	На тему: {\itshape «Венгерский метод решения задачи о назначениях»} \\
	Вариант 6

	\vfill

	\begin{tabular}{p{0.52\textwidth} c c c}
		Студент \underline{ИУ7-13М}            & \underline{\hspace{35mm}}     & \underline{Керимов~А.~Ш.}     \\ [-0.6em]
		{\hspace{23.5mm} \scriptsize (Группа)} & {\scriptsize (Подпись, дата)} & {\scriptsize (Фамилия~И.~О.)} \\
		Преподаватель                          & \underline{\hspace{35mm}}     & \underline{Власов~П.~А.}      \\ [-0.6em]
		                                       & {\scriptsize (Подпись, дата)} & {\scriptsize (Фамилия~И.~О.)} \\
	\end{tabular}

	\vspace{15mm}

	Москва {\the\year}
\end{titlepage}

\setcounter{page}{2}

%\include{01-list-of-performers}
%\include{02-abstract}
%% ГОСТ 7.32-2017
%% 5.4 Содержание
%%
%% 5.4.1 Содержание включает введение, наименование всех разделов и подразделов, пунктов (если они имеют наименование), заключение, список использованных источников и наименования приложений с указанием номеров страниц, с которых начинаются эти элементы отчёта о НИР.
%%
%% В элементе "СОДЕРЖАНИЕ" приводят наименования структурных элементов работы, порядковые номера и заголовки разделов, подразделов (при необходимости — пунктов) основной части работы, обозначения и заголовки её приложений (при наличии приложений).
%% После заголовка каждого элемента ставят отточие и приводят номер страницы работы, на которой начинается данный структурный элемент.
%%
%% Обозначения подразделов приводят после абзацного отступа, равного двум знакам, относительно обозначения разделов.
%% Обозначения пунктов приводят после абзацного отступа, равного четырём знакам относительно обозначения разделов.
%%
%% При необходимости продолжение записи заголовка раздела, подраздела или пункта на второй (последующей) строке выполняют, начиная от уровня начала этого заголовка на первой строке, а продолжение записи заголовка приложения — от уровня записи обозначения этого приложения.
%%
%% 5.4.2 При составлении отчёта, состоящего из двух и более книг, в каждой из них должно быть приведено своё содержание.
%% При этом в первой книге помещают содержание всего отчёта с указанием номеров книг, в последующих — только содержание соответствующей книги.
%% Допускается в первой книге вместо содержания последующих книг указывать только их наименования.
%%
%% 5.4.3 Для отчёта о НИР объёмом не более 10 страниц содержание допускается не составлять.
%%
%% 5.4.4 Содержание следует оформлять в соответствии с 6.13.

\tableofcontents

%\include{04-terms-and-definitions}
%\include{05-list-of-abbreviations-and-symbols}
%% ГОСТ 7.32-2017
%% 5.7 Введение
%%
%% 5.7.1 Введение должно содержать оценку современного состояния решаемой научно-технической проблемы, основание и исходные данные для разработки темы, обоснование необходимости проведения НИР, сведения о планируемом научно-техническом уровне разработки, о патентных исследованиях и выводы из них, сведения о метрологическом обеспечении НИР.
%% Во введении должны быть отражены актуальность и новизна темы, связь данной работы с другими научно-исследовательскими работами.
%%
%% 5.7.2 Во введении промежуточного отчёта по этапу НИР должны быть указаны цели и задачи исследований, выполненных на данном этапе, их место в выполнении отчёта о НИР в целом.
%%
%% 5.7.3 Во введении заключительного отчёта о НИР приводят перечень наименований всех подготовленных промежуточных отчётов по этапам и их регистрационные номера, если они были представлены в соответствующий орган [1] для регистрации.
%%   [1] В Российской Федерации — ЦИТиС, который присваивает эти номера при представлении промежуточного отчёта на регистрацию.


%% Методические указания к выполнению, оформлению и защите выпускной квалификационной работы бакалавра
%% 2.3 Введение
%%
%% Во введении обосновывается актуальность выбранной темы (со ссылками на монографии, научные статьи), формулируется цель работы («Целью работы является...») и перечисляются задачи, которые необходимо решить для достижения этой цели («Для достижения поставленной цели необходимо решить следующие задачи...»)
%%
%% Среди задач, как правило, выделяют аналитические, конструкторские, технологические и исследовательские.
%% Решение этих задач описывается в соответствующих разделах.
%%
%% Рекомендуемый объём введения 2—3 страницы.


\StructuralElement{Введение}

\textbf{Цель работы:} изучение венгерского метода решения задачи о назначениях.

\textbf{Содержание работы}
\begin{enumerate}
	\item реализовать венгерский метод решения задачи о назначениях в виде программы на ЭВМ\footnotemark;
	\item провести решение задачи с матрицей стоимостей, заданной в индивидуальном варианте, рассмотрев два случая:
	\begin{enumerate}
		\item задача о назначениях является задачей минимизации,
		\item задача о назначениях является задачей максимизации.
	\end{enumerate}
\end{enumerate}

\footnotetext{В программе необходимо предусмотреть два режима работы: «итоговый», когда программа печатает только матрицу назначений, и «отладочный», когда на каждой итерации на экран выводится текущая матрица эквивалентной задачи с отмеченной (например, цветом или шрифтом) системой независимых нулей.}


\begin{MainPart}
%% Методические указания к выполнению, оформлению и защите выпускной квалификационной работы бакалавра
%% 2.4 Аналитический раздел
%%
%% В данном разделе расчётно-пояснительной записки проводится анализ предметной области и выделяется основной объект исследования.
%% Если формализовать предметную область с помощью математической модели не удаётся и при этом она сложна для понимания, то для отображения происходящих в ней процессов необходимо использовать методологию IDEF0, а для описания сущностей предметной области и взаимосвязей между ними — ER-модель.
%%
%% Затем выполняется обзор существующих методов и алгоритмов решения идентифицированной проблемы предметной области (опять же с обязательными ссылками на научные источники: монографии, статьи и др.) и их программных реализаций (при наличии), анализируются достоинства и недостатки каждого из них.
%% Выполненный обзор должен позволить объективно оценить актуальное состояние изучаемой проблемы.
%% Результаты проведённого анализа по возможности классифицируются и оформляются в табличной форме.
%%
%% На основе выполненного анализа обосновывается необходимость разработки нового или адаптации существующего метода или алгоритма.
%%
%% Если же целью анализа являлся отбор (на основе чётко сформулированных критериев) тех методов и алгоритмов, которые наиболее эффективно решают поставленную задачу, то форма представления результата должна подтвердить обоснованность сделанного выбора, в том числе — полноту и корректность предложенных автором критериев отбора.
%%
%% Одним из основных выводов аналитического раздела должно стать формализованное описание проблемы предметной области, на решение которой будет направлен данный проект, включающее в себя:
%% — описание входных и выходных данных;
%% — указание ограничений, в рамках которых будет разработан новый, адаптирован существующий или просто реализован метод или алгоритм;
%% — описание критериев сравнения нескольких реализаций метода или алгоритма;
%% — описание способов тестирования разработанного, адаптированного или реализованного метода или алгоритма;
%% — описание функциональных требований к разрабатываемому программному обеспечению,
%% при этом в зависимости от направления работы отдельные пункты могут отсутствовать.
%%
%% Если в результате работы будет создано программное обеспечение, реализующее большое количество типичных способов взаимодействия с пользователем, необходимо каждый из этих способов описать с помощью диаграммы прецедентов [4, 5].
%%
%% Рекомендуемый объём аналитического раздела 25—30 страниц.


\chapter{Аналитический раздел}

\section{Постановка задачи}

\subsection{Содержательная постановка}

В распоряжении работодателя имеется $n$ работ и $n$ исполнителей.
Стоимость выполнения $i$-й работы $j$-м исполнителем составляет $c_{ij} \geqslant 0$ единиц.

\begin{itemize}
	\item Требуется распределить все работы по исполнителям так, чтобы каждый исполнитель выполнял ровно 1 работу.
	\item Общая стоимость всех работ должна быть минимальной.
\end{itemize}

\subsection{Математическая постановка}

Обозначим за матрицу стоимостей
\begin{equation}
	C = (c_{ij}), \quad i, j = \overline{1, n}.
\end{equation}

Введём так называемые управляемые переменные:
\begin{equation}
	x_{ij} = \begin{cases}
		1, \text{ если } i \text{-ю работу выполняет } j \text{-й работник}; \\
		0, \text{ иначе}.
	\end{cases}
	i, j = \overline{1, n}
\end{equation}

Обозначим за матрицу назначений
\begin{equation}
	X = (x_{ij}), \quad  i, j = \overline{1, n}.
\end{equation}

Математическая постановка задачи о назначениях:
\begin{equation}
	\begin{dcases}
		f(x) = \sum_{i=1}^{n} \sum_{j=1}^n c_{ij}x_{ij} & \to \min, \\
		\sum_{i=1}^{n} x_{ij} = 1, & j = \overline{1, n}, \\
		\sum_{j=1}^{n} x_{ij} = 1, & i = \overline{1, n}, \\
		x_{ij} \in \{0, 1\}, & i, j = \overline{1, n}. \\
	\end{dcases}
\end{equation}

\section{Исходные данные варианта 6}

\begingroup
\renewcommand*{\arraystretch}{0.6}
\begin{equation}
	C = \begin{bmatrix}
		10 &  8 &  6 &  4 &  9 \\
		11 &  9 & 10 &  5 &  6 \\
		 5 & 10 &  8 &  6 &  4 \\
		 3 & 11 &  9 &  6 &  6 \\
		 8 & 10 & 11 &  8 &  7 \\
	\end{bmatrix}
\end{equation}
\endgroup

%% Методические указания к выполнению, оформлению и защите выпускной квалификационной работы бакалавра
%% 2.5 Конструкторский раздел
%%
%% В конструкторском разделе описывается разрабатываемый и/или модифицируемый метод или алгоритм.
%%
%% В случае если в бакалаврском проекте разрабатывается новый метод или алгоритм, необходимо подробно изложить их суть, привести всё необходимые для их реализации математические выкладки, обосновать последовательность этапов выполнения.
%% При этом для каждого этапа следует выделить необходимые исходные данные и получаемые результаты.
%%
%% При использовании известного алгоритма следует указать специфические особенности его практической реализации, присущие решаемой задаче, и пути их решения в ходе программирования.
%% Для описания метода или алгоритма необходимо выбрать наиболее подходящую форму записи (схема (ГОСТ 19.701-90), диаграмма деятельности, псевдокод и т. п.).
%% Учитывая, что на эффективность алгоритма непосредственно влияют используемые структуры данных, в данном разделе РПЗ целесообразно провести сравнительный анализ структур, которые могут быть применены в рамках программной реализации выбранного алгоритма, и обосновать выбор одной из них.
%% В конце описания разработанного и/или модифицируемого алгоритма должны быть приведены выбранные способы тестирования и сами тесты.
%%
%% Перед формированием тестовых наборов данных целесообразно указать выделенные классы эквивалентности.
%% В данной части расчётно-пояснительной записки могут также выполняться расчёты для определения объёмов памяти, необходимой для хранения данных, промежуточных и окончательных результатов работы программы, а также расчёты, позволяющие оценить время решения задачи на ЭВМ.
%% Эти результаты могут использоваться для обоснования правильности выбора метода и/или алгоритма из имеющихся альтернативных вариантов, а также для оценки возможности практически реализовать поставленную задачу на имеющейся технической базе.
%%
%% Другой важный момент, который должен найти своё отражение в конструкторском разделе, это описание структуры разрабатываемого программного обеспечения.
%% Обычно оно включает в себя:
%% — описание общей структуры — определение основных частей (компонентов) и их взаимосвязей по управлению и по данным;
%% — декомпозицию компонентов и построение структурных иерархий;
%% — проектирование компонентов.
%%
%% Для графического представления такого описания, если есть необходимость, следует использовать:
%% — функциональную модель IDEF0 с декомпозицией решения исходной задачи на несколько уровней (разрабатываемые модули обычно играют роль механизмов);
%% — спецификации компонентов (процессов);
%% — модель данных (ER-диаграмма);
%% — диаграмму классов;
%% — диаграмму компонентов;
%% — диаграмму переходов состояний (конечный автомат), характеризующих поведение системы во времени.
%%
%% Рекомендуемый объем конструкторского раздела 25—30 страниц.


\chapter{Конструкторский раздел}

%% Методические указания к выполнению, оформлению и защите выпускной квалификационной работы бакалавра
%% 2.6 Технологический раздел
%%
%% Технологический раздел содержит обоснованный выбор средств программной реализации, описание основных (нетривиальных) моментов разработки и методики тестирования созданного программного обеспечения.
%%
%% В этом же разделе описывается информация, необходимая для сборки и запуска разработанного программного обеспечения, форматы входных, выходных и конфигурационных файлов (если такие имеются), а также интерфейс пользователя и руководство пользователя.
%%
%% Если для правильного функционирования разработанного программного обеспечения требуется некоторая инфраструктура (веб-приложение, база данных, серверное приложение), уместно представить её с помощью диаграммы развёртывания UML.
%%
%% Как уже говорилось, часть технологического раздела должна быть посвящена тестированию разработанного программного обеспечения.
%%
%% Модульное тестирование описывается в технологическом разделе.
%%
%% Системное тестирование может быть описано в технологическом или экспериментальном разделах, в зависимости от глубины его реализации и тематики бакалаврской работы.
%%
%% При проведении тестирования разработанного программного обеспечения следует широко использовать специализированные программные приложения: различные статические анализаторы кода (например, clang); для тестирования утечек памяти в языках программирования, где отсутствует автоматическая «сборка мусора», Valgrind, Doctor Memory и их аналоги, и т. п.
%%
%% Рекомендуемый объём технологического раздела 20—25 страниц.


\chapter{Технологический раздел}

\section{Листинг программы}

\lstinputlisting[
	caption={\code{lab01.m}},
	label={lst:lab01m},
]{../lab01.m}

%% Методические указания к выполнению, оформлению и защите выпускной квалификационной работы бакалавра
%% 2.7 Экспериментальный раздел
%%
%% Данный раздел содержит описание проведённых экспериментов и их результаты.
%% Должно быть обязательно указано, какую цель ставил перед собой автор работы при планировании экспериментов, какие предположения/гипотезы он надеялся подтвердить и/или опровергнуть с их помощью.
%% Результаты оформляются в виде графиков, диаграмм и/или таблиц.
%%
%% Здесь же может быть проведено качественное и количественное сравнение с аналогами.
%%
%% Рекомендуемый объем экспериментального раздела 10—15 страниц.


\chapter{Исследовательский раздел}

\section{Результаты расчётов для задач из индивидуального варианта}

\subsection{Минимизация}

\begin{verbatim}
	Оптимальное решение: X* =
	     0     0     1     0     0
	     0     0     0     1     0
	     0     0     0     0     1
	     1     0     0     0     0
	     0     1     0     0     0

	f* = 28
\end{verbatim}

\subsection{Максимизация}

\begin{verbatim}
	Оптимальное решение: X* =
	     0     0     0     0     1
	     1     0     0     0     0
	     0     0     0     1     0
	     0     1     0     0     0
	     0     0     1     0     0

	f* = 48
\end{verbatim}

\end{MainPart}

%%% ГОСТ 7.32-2017
%% 5.9 Заключение
%%
%% Заключение должно содержать:
%% - краткие выводы по результатам выполненной НИР или отдельных её этапов;
%% - оценку полноты решений поставленных задач;
%% - разработку рекомендаций и исходных данных по конкретному использованию результатов НИР;
%% - результаты оценки технико-экономической эффективности внедрения;
%% - результаты оценки научно-технического уровня выполненной НИР в сравнении с лучшими достижениями в этой области.

%% Методические указания к выполнению, оформлению и защите выпускной квалификационной работы бакалавра
%% 2.10 Заключение
%%
%% Заключение содержит краткие выводы по всей работе и оценку полноты решения поставленной задачи.

\StructuralElement{Заключение}

%\include{91-references}
%\include{92-attachments}

\end{document}
