%% Методические указания к выполнению, оформлению и защите выпускной квалификационной работы бакалавра
%% 2.7 Экспериментальный раздел
%%
%% Данный раздел содержит описание проведённых экспериментов и их результаты.
%% Должно быть обязательно указано, какую цель ставил перед собой автор работы при планировании экспериментов, какие предположения/гипотезы он надеялся подтвердить и/или опровергнуть с их помощью.
%% Результаты оформляются в виде графиков, диаграмм и/или таблиц.
%%
%% Здесь же может быть проведено качественное и количественное сравнение с аналогами.
%%
%% Рекомендуемый объем экспериментального раздела 10—15 страниц.


\chapter{Исследовательский раздел}

\section{Результаты расчётов для задач из индивидуального варианта}

\subsection{Минимизация}

\begin{verbatim}
	Оптимальное решение: X* =
	     0     0     1     0     0
	     0     0     0     1     0
	     0     0     0     0     1
	     1     0     0     0     0
	     0     1     0     0     0

	f* = 28
\end{verbatim}

\subsection{Максимизация}

\begin{verbatim}
	Оптимальное решение: X* =
	     0     0     0     0     1
	     1     0     0     0     0
	     0     0     0     1     0
	     0     1     0     0     0
	     0     0     1     0     0

	f* = 48
\end{verbatim}
