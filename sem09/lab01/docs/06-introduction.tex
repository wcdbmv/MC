%% ГОСТ 7.32-2017
%% 5.7 Введение
%%
%% 5.7.1 Введение должно содержать оценку современного состояния решаемой научно-технической проблемы, основание и исходные данные для разработки темы, обоснование необходимости проведения НИР, сведения о планируемом научно-техническом уровне разработки, о патентных исследованиях и выводы из них, сведения о метрологическом обеспечении НИР.
%% Во введении должны быть отражены актуальность и новизна темы, связь данной работы с другими научно-исследовательскими работами.
%%
%% 5.7.2 Во введении промежуточного отчёта по этапу НИР должны быть указаны цели и задачи исследований, выполненных на данном этапе, их место в выполнении отчёта о НИР в целом.
%%
%% 5.7.3 Во введении заключительного отчёта о НИР приводят перечень наименований всех подготовленных промежуточных отчётов по этапам и их регистрационные номера, если они были представлены в соответствующий орган [1] для регистрации.
%%   [1] В Российской Федерации — ЦИТиС, который присваивает эти номера при представлении промежуточного отчёта на регистрацию.


%% Методические указания к выполнению, оформлению и защите выпускной квалификационной работы бакалавра
%% 2.3 Введение
%%
%% Во введении обосновывается актуальность выбранной темы (со ссылками на монографии, научные статьи), формулируется цель работы («Целью работы является...») и перечисляются задачи, которые необходимо решить для достижения этой цели («Для достижения поставленной цели необходимо решить следующие задачи...»)
%%
%% Среди задач, как правило, выделяют аналитические, конструкторские, технологические и исследовательские.
%% Решение этих задач описывается в соответствующих разделах.
%%
%% Рекомендуемый объём введения 2—3 страницы.


\StructuralElement{Введение}

\textbf{Цель работы:} изучение венгерского метода решения задачи о назначениях.

\textbf{Содержание работы}
\begin{enumerate}
	\item реализовать венгерский метод решения задачи о назначениях в виде программы на ЭВМ\footnotemark;
	\item провести решение задачи с матрицей стоимостей, заданной в индивидуальном варианте, рассмотрев два случая:
	\begin{enumerate}
		\item задача о назначениях является задачей минимизации,
		\item задача о назначениях является задачей максимизации.
	\end{enumerate}
\end{enumerate}

\footnotetext{В программе необходимо предусмотреть два режима работы: «итоговый», когда программа печатает только матрицу назначений, и «отладочный», когда на каждой итерации на экран выводится текущая матрица эквивалентной задачи с отмеченной (например, цветом или шрифтом) системой независимых нулей.}
